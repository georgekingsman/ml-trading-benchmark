\documentclass[12pt,a4paper]{article}
\usepackage{CJKutf8}
\usepackage[margin=2.2cm]{geometry}
\usepackage{setspace}
\usepackage{booktabs}
\usepackage{longtable}
\usepackage{array}
\usepackage{enumitem}
\usepackage{xcolor}
\usepackage{hyperref}
\hypersetup{colorlinks=true,linkcolor=black,urlcolor=blue,citecolor=black}
\setstretch{1.25}

\title{AI Hackathon 项目完整说明书\\\large ML Trading Benchmark:可复现的机器学习量化交易评测系统}
\author{项目团队:AutoSurvey / Benchmark 组}
\date{\today}

\begin{document}
\begin{CJK*}{UTF8}{gbsn}
\maketitle

\section{一、产品简介(Product Overview)}

\subsection{1.1 项目定位}
本项目是一个面向\textbf{量化交易 AI 模型评测}的工程化基准系统(Benchmark Toolkit),核心目标是:
\begin{itemize}[leftmargin=2em]
    \item 在统一协议下公平比较不同机器学习模型在真实交易场景中的表现;
    \item 显式纳入\textbf{交易成本、市场状态切换、数据泄漏控制}等现实因素;
    \item 产出可直接用于论文、竞赛答辩和工程复现的标准化结果(表格+图+JSON)。
\end{itemize}

\subsection{1.2 待解决问题}
传统量化 AI 评估常见问题包括:
\begin{itemize}[leftmargin=2em]
    \item 仅比较“模型彼此之间”,缺少对被动基准(如 SPY Buy-and-Hold)的对照;
    \item 忽视交易成本后“纸面收益高、实盘收益差”;
    \item 存在前瞻偏差、标签泄漏,导致结果不可落地;
    \item 不同论文评估协议不一致,结论难以横向比较。
\end{itemize}

本系统以\textbf{评估协议标准化}为抓手,解决上述问题。

\subsection{1.3 应用场景}
\begin{itemize}[leftmargin=2em]
    \item AI 量化策略研发团队:用于模型筛选与回测基线建设;
    \item 高校/科研团队:用于可复现实验与论文实验章节;
    \item 竞赛评审场景:用于直观展示“模型效果 + 工程严谨性 + 可落地性”。
\end{itemize}

\section{二、核心功能(Core Functions)}

\subsection{2.1 一站式流水线}
系统支持从数据到报告的全自动流程:
\begin{enumerate}[leftmargin=2em]
    \item 数据下载与清洗(50只美股ETF,约20年日频);
    \item 特征工程(13个技术特征);
    \item Walk-forward 训练/验证/测试划分(含5天embargo);
    \item 9类模型训练与预测(传统ML + 深度学习 + 策略基线);
    \item 回测执行(Top-K多空、可配置调仓与成本);
    \item 指标计算(CAGR、Sharpe、Max Drawdown、IC/ICIR等);
    \item 自动生成图表、LaTeX表格、JSON结果文件。
\end{enumerate}

\subsection{2.2 可配置实验协议}
通过配置文件统一管理关键参数:
\begin{itemize}[leftmargin=2em]
    \item 数据范围、任务定义(1/5/20日收益预测);
    \item 数据切分与embargo长度;
    \item 启用模型与超参数;
    \item 回测规则(Top-K、调仓频率、成本场景);
    \item 指标集合与报告输出格式。
\end{itemize}

\subsection{2.3 结果可复现与可审计}
\begin{itemize}[leftmargin=2em]
    \item 全流程脚本化执行,支持\texttt{--skip-download}、\texttt{--skip-features}加速复跑;
    \item 输出结构化文件(CSV/TeX/PDF/JSON),便于评审抽查;
    \item 固定协议后可重复获得同类结论,降低“偶然性结果”风险。
\end{itemize}

\section{三、产品亮点(Highlights)}

\subsection{3.1 亮点一:真实交易约束下的评估}
系统将交易成本设置为0/5/10/15/25 bps多场景评估,显式考察“策略对摩擦成本的脆弱性”,避免仅在无成本假设下得出乐观结论。

\subsection{3.2 亮点二:严格防止数据泄漏}
采用\textbf{rolling标准化 + walk-forward + embargo}的组合:
\begin{itemize}[leftmargin=2em]
    \item 仅使用历史窗口参数标准化,杜绝未来信息;
    \item 训练/验证/测试按时间前向推进;
    \item 通过embargo防止标签时间重叠。
\end{itemize}

\subsection{3.3 亮点三:不仅看收益,还看信号质量}
除了策略层指标(Sharpe/CAGR/MaxDD),还计算\textbf{IC/ICIR}评估预测信号与未来收益的横截面相关性,更精准定位“模型弱在预测还是弱在执行”。

\subsection{3.4 亮点四:完整实验矩阵}
项目包含主结果、成本敏感性、市场状态分解、长短仓对比、特征重要性、集成模型、调仓频率敏感度、Top-K敏感度、统计检验等多维实验,覆盖“性能、稳定性、鲁棒性、统计显著性”四个维度。

\subsection{3.5 亮点五:结果对外表达友好}
系统自动生成可直接用于论文与答辩的图表和表格,显著降低整理工作量,提升团队产出效率。

\section{四、技术方案特色(Technical Design)}

\subsection{4.1 总体架构}
\begin{center}
\begin{tabular}{p{0.18\linewidth}p{0.74\linewidth}}
\toprule
模块 & 功能说明 \\
\midrule
数据层 & 获取并清洗ETF OHLCV数据,构建统一时间索引与数据字典 \\
特征层 & 构造13个技术因子并进行严格历史窗口标准化 \\
建模层 & 统一\texttt{fit/predict}接口,支持9类模型并行评估 \\
策略层 & Top-K多空组合构建、调仓执行、成本扣减 \\
评估层 & 绩效指标、IC/ICIR、置信区间、统计检验 \\
报告层 & 自动输出CSV/LaTeX/PDF/JSON结果与可视化图表 \\
\bottomrule
\end{tabular}
\end{center}

\subsection{4.2 数据与特征方案}
\begin{itemize}[leftmargin=2em]
    \item 资产池:50只美国ETF,覆盖权益、固收、商品、货币多资产类别;
    \item 时间跨度:2005-01 至 2024-12(约20年);
    \item 特征集合:收益率、波动率、动量、RSI、均线偏离、量价关系等13项;
    \item 标准化:252日滚动z-score(严格回溯窗口)。
\end{itemize}

\subsection{4.3 模型与策略方案}
\textbf{模型池(9类)}:
\begin{itemize}[leftmargin=2em]
    \item 传统机器学习:Linear/Ridge/Logistic Regression、Random Forest、LightGBM;
    \item 深度学习:MLP、LSTM;
    \item 策略基线:Momentum、Mean Reversion。
\end{itemize}

\textbf{默认交易策略}:
\begin{itemize}[leftmargin=2em]
    \item 每5个交易日调仓;
    \item 预测排序后做多前10、做空后10(等权);
    \item 成本模型:\(\text{cost}=(\text{fee}+\text{slippage})\times \sum |\Delta w|\)。
\end{itemize}

\subsection{4.4 评估与统计方案}
\begin{itemize}[leftmargin=2em]
    \item 绩效指标:CAGR、Sharpe、MaxDD、Calmar、Turnover、Hit Rate;
    \item 信号指标:IC、ICIR;
    \item 置信度分析:Bootstrap置信区间;
    \item 显著性分析:Diebold-Mariano检验 + BH-FDR多重比较校正。
\end{itemize}

\subsection{4.5 工程可复现性方案}
\begin{itemize}[leftmargin=2em]
    \item 单命令全流程执行(\texttt{run\_all.py} / \texttt{run\_robustness.py});
    \item 配置驱动实验,避免“脚本散落式”不可维护问题;
    \item 统一目录结构输出,支持复查、复跑与二次分析。
\end{itemize}

\section{五、阶段性结果与价值(Evidence \& Value)}

\subsection{5.1 阶段性结果(示例)}
基于当前实验文档和报告输出,系统已形成完整主结果表与鲁棒性分析表。主要观察包括:
\begin{itemize}[leftmargin=2em]
    \item 在无成本条件下,部分ML模型可获得正收益,但整体Sharpe仍显著受限;
    \item 在15bps成本下,多空策略普遍出现绩效明显下滑,证明成本建模必要性;
    \item IC/ICIR整体偏低,说明问题核心在于信号弱而非单一回测参数设置。
\end{itemize}

\subsection{5.2 项目价值}
\begin{itemize}[leftmargin=2em]
    \item \textbf{对评委可理解:} 结构化流程+可视化输出,便于快速理解方法有效性;
    \item \textbf{对团队可落地:} 可直接扩展新模型/新市场/新频率,形成持续评估平台;
    \item \textbf{对行业有意义:} 强调“严谨评估优先于单次高收益展示”,有助于减少过拟合叙事。
\end{itemize}

\section{六、未来规划(Roadmap)}

\begin{itemize}[leftmargin=2em]
    \item 接入更多数据模态:基本面、新闻文本、宏观因子;
    \item 引入在线学习与时变参数模型,提高市场切换适应性;
    \item 增加风险约束(行业暴露/波动目标/换手上限)与组合优化模块;
    \item 开放标准任务集,建设可持续对比榜单(Leaderboard)。
\end{itemize}

\section{七、结语}
本项目不仅是“模型跑分工具”,更是面向真实交易环境的\textbf{评估基础设施}。我们希望通过可复现、可解释、可审计的实验范式,为AI在量化交易领域的工程落地提供一套可复制的方法论。

\vspace{1em}
\noindent\textbf{附:建议答辩口径(30秒)}\\
我们做的不是“再发明一个模型”,而是搭建了一套可复现、可审计、面向真实成本约束的AI量化评测系统。它把数据、特征、训练、回测、统计检验和报告生成打通成标准化流水线,能够快速回答“这个模型在真实世界是否有效”这一核心问题。

\end{CJK*}
\end{document}